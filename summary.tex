\documentclass{article}

\title{Topic Proposal Summary: \\ Distributed Version Control Systems}
\author{
	Olof Johansson  \textless olof@ethup.se\textgreater \\
	Daniel Persson \textless daniel@silvertejp.org\textgreater
}

\begin{document}

	\maketitle
	
	Our proposed topic for the thesis is distributed version control (DVCS) 
	and feasible workflows relating to this. Version control and source control
	management (SCM) is an important task in software engineering, and the 
	tools used determines if it will become an obstruction rather than the 
	convenience that it probably was intended to be. One of the current 
	trends within source control management is to have the version control 
	system be distributed\cite{sink10}, without a central authority (at 
	least not a central authority imposed by the system itself). 
	
	According to a survey among users of the Eclipse IDE\cite{eclipse10}, the 
	rise of one such system --- git --- has increased from 2.4\% to 6.8\% from 
	2009 to 2010. And it is not the only DVCS with a steep increase in usage: 
	Mercurial increased from 1.1\% to 3\% over the same period of time. 
	DVCS tools are especially popular within the open source community, 
	with large projects, such as the Linux kernel\cite{kernel-git}, 
	Gnome\cite{gnome-git} and Firefox\cite{firefox-hg} using them.

	It is not unusual to use distributed version control system in a 
	centralized way, but the opportunity to use it decentralized is 
	interesting. Within our project we are trying to enforce a distributed 
	workflow, and it would be interesting to learn what the benefits are, 
	what are the obstacles, and how we are going to implement this. It is
	important to evaluate the benefit from this workflow and document the
	experience gained from its use in the project.

	\bibliographystyle{plain}
	\bibliography{references}

\end{document}
