\documentclass{article}

\title{Distributed Version Control Systems}

%\author{
% \IEEEauthorblockN{Daniel Persson}
% \IEEEauthorblockA{daniel@silvertejp.org} \and
%
% \IEEEauthorblockN{Olof Johansson}
% \IEEEauthorblockA{olof@ethup.se}
%}
%
%\IEEEpubid{0000--0000/00\$00~\copyright~2011}
%
\begin{document}

\maketitle

\begin{abstract}
 Place abstract here.
\end{abstract}

%\begin{IEEEkeywords}
% Version Control, Configuration Management
%\end{IEEEkeywords}

\section{Introduction}

%\section{Background}
%
%\subsection{The History of Version Control}
%The history of Version Control Systems can be divided into three
%categories; the ones that only manages local files, the ones that rely
%on a central server to serve connecting clients, and finally the
%distributed ones, where every 'node' can act as both a client and a
%server. The three categories roughly follow each other
%chronologically, naturally with transition-periods in between. At the
%moment the dominating tools are mostly client-server, but the
%distributed ones are rising fast in popularity, so perhaps we are
%currently in the start of a new transition.
%
%\subsubsection{The Local Era}
%The first VCS ever released was the \emph{Source Code Control System}
%(SCCS) in 1972. SCCS tried to solve a number of problems that software
%developers of that time had. Manually managing several versions of the
%same product simultaneously was not feasible in the long run for
%several reasons described by the creator of SCCS, Marc J. Rochkind
%\cite{sccs}:
%
%\begin{itemize}
% \item The amount of space to store the source code may be several
%       times that needed for any particular version.
% \item Fixes made to one version sometimes fail to get made to
%       other versions.
% \item When changes occur it is difficult to tell exactly what changed
%       and when.
% \item When a customer has a problem it is hard to figure out what
%       version he has.
%\end{itemize}
%
%Instead of saving entire files in various states, SCCS stores the
%differences between versions of the same file (called diffs, or
%deltas). With each delta SCCS also stores metadata such as who made
%the change, why, and when. This did not only save precious storage
%space, but also provided the \emph{traceability} that the software
%developers had previously lacked.
%
%It is worth noting that SCCS was not the only VCS that followed the
%philosophy of storing the files locally, \emph{Revision Control
%  System} (RCS) was released in 1982 and gradually took over as the
%dominant VCS for Unix.
%
\section{Research Methodology}

\subsection{Research Questions}
Given the description in the background we intend to investigate the
benefits and drawbacks of using that workflow. To tackle these issues
we have identified the following research questions:

\begin{itemize}
 \item How do the developers adapt to the workflow described above?
 \item How does the workflow affect code quality in relation to the
       release management and code review?
\end{itemize}

By code quality we mean both adherence to the Coding Convention and
error frequency, specifically build and automatic unit-test errors.

\subsection{Research Methodology}
To answer the questions we will conduct both a review of literature in
the field of configuration management in general, and version control
systems in particular, and then conduct a post-mortem analysis of the
project. The primary means of doing the post-mortem analysis will be
to conduct interviews with participants in their various roles. In
addition to this we will also collect data from various project
management and configuration management tools and systems.

\subsubsection{Literature Review}


\subsubsection{Post-mortem Analysis}
The post-mortem analysis will primarily be based on interviews with
participants in the Telia Smart Home project and questionnaries with
members from other projects within the same course framework. We will
base the interviews and questionnaries around these questions:

\begin{itemize}
 \item How would you rate your previous experience in using Version Control 
       Systems?
 \item How would you rate your proficiency in using Version Control Systems?
 \item What benefits do you see in using this workflow?
 \item What drawbacks do you see in using this workflow?
 \item Do you have any suggestions for improvements?
 \item What is your overall opinion of the workflow?
\end{itemize}

The post-mortem analysis will also include an analysis of various data
collected during the project. Most of this data will come from
different configuration and project management tools, such as Git for
Version Control, Jenkins for continuous integration and Redmine for
time management and issue tracking.

From Git we will extract metadata related to each commit, such as
time, size, author and commit messages, Jenkins can produce data on
build and test errors, and from Redmine we will compile reports
related to time spent on release management.

The data collected from the project and configuration management tools
will then be correlated with the data gained from the interviews and
questionnaries.

\bibliographystyle{plain} 
\bibliography{references}

\end{document}
